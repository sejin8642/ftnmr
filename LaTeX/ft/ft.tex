\documentclass[letterpaper, 11pt]{article}
\usepackage[inner = 1.5 cm, outer = 2.5 cm, top = 2.0cm, bottom = 2.0cm, bindingoffset = 0.0 cm]{geometry}
\usepackage{amsmath}
\usepackage{lipsum}
\usepackage{makeidx}
\makeindex

% List of macros
\newcommand{\fourier}[2]{\mathcal{F}_{#1}[#2]}
\newcommand{\fint}{\int_{-\infty}^{\infty}}

% Numbering equations with section number
\numberwithin{equation}{section}

\begin{document}
% Front cover page with title and name
\begin{titlepage}
	\begin{center}
		% \vfill must have preceding text line
		\Huge Title goes here \vfill 
	\end{center}

	\begin{flushright}
		Sejin Nam\\
		University of Hawaii at Manoa
	\end{flushright}
\end{titlepage}

% Table of Contents
\pagenumbering{roman}
\tableofcontents
\clearpage

% First Section: prerequisite mathematics
\pagenumbering{arabic}
\section{Prerequisite Mathematics}
A few prerequisite mathematics will be discussed here in anticipation of their requirements in the subsequent sections after this section. 

% 1.1 Dirac Delta Function
\subsection{Dirac Delta Function}\index{Dirac Delta function}
Dirac Delta function, \(\delta (t)\), has the following property:
\begin{align}
	\delta (t)	&=\begin{cases}
		\infty, & \text{if } t = 0 \\
		0,	& \text{otherwise}
	\end{cases}\\
		1	&= \fint \delta (t) dt
\end{align}
(In my discussion of function, \(t\) variable represents time physicall). The following integral is also a Dirac Delta function:
\begin{align}
	\delta (t) = \fint e^{2\pi ift} df\label{eq:1}
\end{align}
The above relations will be used throughout the discussion of this article.

% Second Section: Fourier transform
\section{Fourier Transform}\index{Fourier transform}
Fourier transform is a mathematical transformation, in the context of my discussion, that maps one function in time domain into another function in frequency domain. There are two types of FT, Fourier transform in short: continuous and discrete. Discrete FT is of great importance in many modern data processing application as modern computers are digital, meaning they perform computation in discrete manner. I'll discuss a continous case first before moving onto the discrete case.

% 2.1 Continuous FT
\subsection{Continuous Fourier Transform}\index{Fourier transform}
Consider a continuous function of \(t\), \(x(t)\). The continuous FT of \(x\) is as follows:
\begin{equation}
	\begin{aligned}[b]
		\fourier{t}{x}	&= \fint x(t)e^{-2\pi ift} dt \\
				&= X(f)
	\end{aligned}
\end{equation}
\(X(f)\) is also continuous in \(f\) (in my discussion \(f\) represents frequency physically). You can also perform inverse FT on \(X\), which is:
\begin{equation}
	\begin{aligned}[b]
		\fourier{f}{X}	&= \fint X(f)e^{2\pi ift} df \\
				&= \fint \fint x(\tau)e^{-2\pi if\tau} d\tau\ e^{2\pi ift} df \\
				&= \fint x(\tau) \fint e^{2\pi if(t - \tau)} df d\tau \\
				&= \fint x(\tau) \delta (t - \tau) d\tau \\
				&= x(t)
	\end{aligned}
\end{equation}
where I used \eqref{eq:1} orthogonality relation.

% 2.1.1 Sinusoidal exponential decay signal
\subsubsection{Sinusoidal exponential decay function}
Sinusoidal exponential decay (I call it SED) function is as follows:
\begin{align}
	s &= s_{0}\cos{(2\pi ft)}e^{-t/T}
\end{align}
You can have sine instead of cosine in the above expression. For our discussion, SED of choice is combination of sine and cosine exponential decays:
\begin{align}
	s &= s_{0}e^{2\pi ift}e^{-t/T}
\end{align}
The real part or the imaginary part can represent a real physical signal.
\printindex
\end{document}
