\documentclass[letterpaper, 11pt]{article}
\usepackage[inner = 1.5 cm, outer = 2.5 cm, top = 2.0cm, bottom = 2.0cm, bindingoffset = 0.0 cm]{geometry}
\usepackage{amsmath}
\usepackage{lipsum}
\usepackage{makeidx}
\makeindex

% List of macros
\newcommand{\fourier}[2]{\mathcal{F}_{#1}[#2]}

% Numbering equations with section number
\numberwithin{equation}{section}

\begin{document}
% Front cover page with title and name
\begin{titlepage}
	\begin{center}
		stupid title\\
		[10cm]
	\end{center}

	\begin{flushright}
		Sejin Nam
	\end{flushright}
\end{titlepage}

% Table of Contents
\pagenumbering{roman}
\tableofcontents
\clearpage

% First Section
\pagenumbering{arabic}
\section{Free Induction Decay}
Here describes what FID\index{FID} is. \lipsum[1]
\subsection{What is FID?}
FID is an acronym for free induction decay

% Second Section
\section{Fourier Transform}
\lipsum[2]
\subsection{Continuous Fourier Transform}
\begin{equation}
	\begin{aligned}[b]
		X(f)	&= \fourier{t}{x}\\
			&= \int_{-\infty}^{\infty}
	\end{aligned}
\end{equation}

\begin{enumerate}
	% first question
	\item Let's see if the pdf \index{stupid file} automatically reloads after building a modified pdf. Please write your lab number at the top right corner of this quiz as well as your signature. On the back of this quiz, please write your name.\vspace{0.25cm}

For questions 2 and 3, consider the following quantity called standard deviation of x:
	\begin{align}
			\sigma_{x} = \sqrt[]{\frac{1}{N - 1} \sum_{i = 1}^{N} (x_{i} - \bar{x})^{2}}
	\end{align}
	
where \(x_{1}, x_{2}, \dots, x_{N}\) are the values of \(x\) for \(N\) measurements. Suppose you made the measurement of \(x\) five times and obtained the results 56, 57, 56, 58, 57 (for convenience, any units were omitted).

	% second question
	\item (16 pts) Calculate \(\sigma_{x}\). Show all your work. 
	
	% third question
	\item (8 pts) Report your final result \(x_{\text{exp}}\) appropriately. 
	
	% third question
	\item (8 pts) Briefly describe what we are doing for today's experiment.   
\end{enumerate}
\printindex
\end{document}
